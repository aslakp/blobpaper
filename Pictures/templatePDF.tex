\documentclass[10pt,twocolumn]{article} 
%\usepackage{simpleConference}
\usepackage{times}
\usepackage{graphicx}
\usepackage{amssymb}
\usepackage{url,hyperref}

\begin{document}

\title{\LaTeX\ Guidelines for Simple, Two-Column Papers}

\author{Edward A. Lee\\
\\
EECS 290N Report\\
September 13, 2004 \\
\\
University of California at Berkeley\\
Berkeley, CA, 94720, USA\\
\\
eal@eecs. berkeley.edu\\
}

\maketitle
\thispagestyle{empty}

\begin{abstract}
   This is a simple sample of a document created using \LaTeX
   (specifically pdflatex)
   that includes a figure from the Vergil visual editor for Ptolemy II
   that was created by printing to the Acrobat Distiller to get a PDF file.
   It also illustrates a simple two-column conference paper style,
   and use of bibtex to handle bibligraphies.
\end{abstract}

%\begin{figure}[!b]
%  \begin{center}
%    \includegraphics[width=3.5in]{figure.pdf}
%  \end{center}

  \caption{\small Figure caption. To get a figure to span two
      columns, use the environment figure* rather than figure.}
  \label{fig-label}
\end{figure}

\section{Using \LaTeX with PDF Figures}

This is a sample document for use with pdflatex, which is
a program that is included with the Miktex distribution
that directly produces PDF files from \LaTeX sources.
To run \LaTeX on this file, you need the following files:
\begin{enumerate}
\item templatePDF.tex (this file)
\item figure.pdf (the figure file)
\item simpleConference.sty (style file)
\item refs.bib (bibiliography file)
\end{enumerate}
\noindent
To create a PDF file, execute the following commands:
\begin{enumerate}
\item pdflatex templatePDF
\item bibtex templatePDF
\item pdflatex templatePDF
\item pdflatex templatePDF
\end{enumerate}
\noindent
Yes (strangely) it is necessary to run pdflatex three times.
The result will be a PDF file (plus several other files that \LaTeX
produces).  You will need a mechanism, of course, for executing
commands on the command line. If you are using Windows, I recommend
installing Cygwin and using its bash shell.

\section{How to Include Vergil Diagrams as Figures}

Suppose you wish to include a figure, like that in figure \ref{fig-label}.
The simplest mechanism is to install Adobe Acrobat, which includes
a ``printer'' called ``Acrobat Distiller.'' Printing to this printer
creates a PDF file, which can be included in a document as shown
here.  To include Ptolemy II models \cite{PtolemyVol1:04},
just print to the distiller from within Vergil and reference
the PDF file in your \LaTeX document.

There is a bit more work to do, however.
The file that is produced by the distiller represents
a complete page, not the individual figure.
You can open it in using Acrobat (version 5.0 or later),
and select Document $\rightarrow$ Crop Pages from the menu.
In the resulting dialog, check ``Remove White Margins.''
Save the modified PDF file in a file and then reference
it in the \LaTeX file as shown in this example.

An alternative is to generate EPS (encapsulated postscript),
but the process is much more complex and fragile.
I recommend using pdflatex and Adobe Acrobat.

\bibliographystyle{abbrv}
\bibliography{refs}
\end{document}
